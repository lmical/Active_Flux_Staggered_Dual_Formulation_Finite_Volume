\documentclass[letter,11pt]{letter}

\usepackage[margin=0.89in]{geometry}
%\usepackage{moreverb,latexsym}
%\usepackage{graphics,graphicx,curves,epic,eepic,ulem,bbm,mathbbol,enumerate,latexsym,multicol}
\usepackage{lipsum}
\usepackage{amssymb}
\usepackage{amsmath}
\usepackage{amsfonts}
\usepackage{graphicx}
\usepackage{caption}
%\usepackage{epstopdf}
%\usepackage{theorem}
\usepackage{mathrsfs}
\usepackage{mathtools}
%\usepackage{empheq}
\usepackage{bm}
\usepackage{bbm}
\usepackage{color}
\usepackage{setspace}
\usepackage{exscale}
\usepackage{relsize}
%\usepackage{float}
\usepackage{picinpar}
\usepackage{extarrows}
\usepackage{multirow}
%\usepackage{cite}
\usepackage[backend=bibtex,style=numeric]{biblatex}
\addbibresource{sn-bibliography.bib}
\defbibheading{bibliography}[\refname]{\par\bigskip\noindent\textbf{#1}\par}

\usepackage{nicefrac}
%%% PACKAGES
\usepackage{booktabs} % for much better looking tables
\usepackage{array} % for better arrays (eg matrices) in maths
\usepackage{paralist}
\usepackage[normalem]{ulem}
\newcommand{\vla}[1]{\textcolor{magenta}{#1}}
\usepackage{tikz}
\usetikzlibrary{positioning}
\usepackage{xcolor}
\usepackage{enumitem,kantlipsum}

%\usepackage{subcaption}
%\usepackage{epsfig}
\usepackage{algorithm}
\usepackage{algpseudocode}

\newcommand{\RIcolor}[1]{{\leavevmode\color{red} #1}}
\newcommand{\RIIcolor}[1]{{\leavevmode\color{blue} #1}}
\newcommand{\RIIIcolor}[1]{{\leavevmode\color{magenta} #1}}
\usepackage{amsthm}
\usepackage{graphicx}
\usepackage{hyperref}  % optional, for clickable references

% Define a custom figure counter for letters
\newcounter{myfigure}
\newcommand{\myfigure}[3][0.8\textwidth]{%
	\refstepcounter{myfigure}%
	\begin{center}
		\includegraphics[width=#1, height=0.75\textheight, keepaspectratio]{#2}%
		\\[0.5em]
		\textbf{Figure~\themyfigure:} #3
	\end{center}%
}


\theoremstyle{remark}
\newtheorem*{remark}{Remark}

\newcommand{\uvec}[2][3]{\bm{#2\mkern-#1mu}\mkern#1mu}




\newcommand*\xbar[1]{%
  \hbox{%
    \vbox{%
      \hrule height 0.5pt % The actual bar
      \kern0.4ex%         % Distance between bar and symbol
      \hbox{%
        \kern-0.05em%      % Shortening on the left side
        \ensuremath{#1}%
        \kern-0.05em%      % Shortening on the right side
      }%
    }%
  }%
}

% Some of the article customisations are relevant for this class

\name{Alexander Kurganov} % To be used for the return address on the envelope
\signature{Alexander Kurganov} % Goes after the closing (ie at the end of the letter, with space for a signature)
\address{Alexander Kurganov\\
Department of Mathematics\\
Southern University of Science and Technology\\
Shenzhen, 518055, China}
% Alternatively, these may be set on an individual basis within each letter environment.

%\makelabels % this command prints envelope labels on the final page of the document

\begin{document}
\begin{letter}{Computers \& Fluids\\
CAF-D-25-00470R1}

\opening{Dear Editor:} % eg Hello.
We would like to thank the reviewers for carefully reading a revised version of our paper. We have made additional changes to the paper
following the reviewers' comments.

Below, we outline our responses to the Reviewers' comments and questions. In the revised manuscript, the changes marked in
{\color{blue}blue} correspond to remarks made by Reviewer II and those marked in {\color{magenta}magenta} correspond to remarks made by
Reviewer III.

\bigskip
\underline{Response to the Comments of Reviewer II}

\begin{itemize}[wide, labelwidth=!, labelindent=0pt]
\item
{\bf Figure 3.1 caption: $x$'s are not the same}

\smallskip
{\sc\underline{Reply}}: Corrected.
\end{itemize}

\bigskip
\underline{Response to the Comments of Reviewer III}
\begin{itemize}[wide, labelwidth=!, labelindent=0pt]
\item
{\bf Other schemes have successfully mixed non-conservative and conservative formulations of hyperbolic systems, e.g.:

\smallskip
Elena Gaburro, Walter Boscheri, Simone Chiocchetti, Mario Ricchiuto: Discontinuous Galerkin schemes for hyperbolic systems in
non-conservative variables: quasi-conservative formulation with subcell finite volume corrections, Computer Methods in Applied Mechanics and
Engineering, Volume 431 (2024)}

\smallskip
{\sc\underline{Reply}}: We have added a reference to this paper [18] in the second paragraph on page 2.

\medskip
\item
{\bf Phil Roe’s active flux method uses a primitive formulation for point updates without having to resort to a path conservative scheme.
Besides, using a path conservative type scheme for a system of equations that originally is homogeneous does not seem natural.}

\smallskip
{\sc\underline{Reply}}: As we have pointed out in Remark 2.2, ``the proposed DF-FV method is not tied to the PCCU scheme and, in principle,
one can numerically solve the nonconservative system (2.2) using an alternative second-order stable numerical method instead. However, the
PCCU scheme seems to be a reasonable choice, thanks to its distinctive feature: once the path has been selected, the resulting method is not
sensitive to a particular choice on the nonconservative formulation; see [8].''

We would like to stress that when the nonconservative terms are present and the solution is discontinuous, it is natural not to neglect the
contribution of these terms at the cell interfaces. This does not help to recover the conservation properties, but definitely helps to 
stabilize the numerical solution and to make it less sensitve to a particular form of the nonconservative systems used.

\medskip
\item
{\bf We would ask the authors to list in their introduction the advantages of this DF-FV scheme, its advantage over other hybridized schemes
and including its future potential. It would be good to have this all together in the beginning.}

\smallskip
{\sc\underline{Reply}}: We have added a passage to point out the future potential of the proposed DF-FV scheme in the third paragraph on
page 3. One of the main differences between our approach and versions of the Active Flux methods, making use of primitive variables for the
point values, is a higher level of decoupling between the conservative and primitive degrees of freedom. All of the upcoming applications,
mentioned in the Introduction in the revised version of the manuscript, exploit such a decoupling with the help of an appropriate
modification of the post-processing. On the other hand, the strong coupling between cell averages and point values characterizing the
Active Flux methods makes it difficult to apply the same ideas in a straightforward way. We, however, prefer not to speculate on this point
in the current paper and to leave this issue for future investigation.

\smallskip
We would also like to point out that there are many other schemes utilizing the primitive formulation of the governing equations. Conducting
a thorough comparison between these schemes and the proposed DF-FV method is out of scope of the current paper.

\medskip
\item
{\bf There is a small typo on p. 4 after (2.6): “eref 1.2”}

\smallskip
{\sc\underline{Reply}}: Corrected.
\end{itemize}

\vspace*{2cm}
\closing{Sincerely} % eg Regards,
\end{letter}

\end{document}

